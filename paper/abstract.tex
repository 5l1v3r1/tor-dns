\begin{abstract}
Previous attacks that link the sender and receiver of traffic in the Tor
network (``correlation attacks'') have generally relied on analyzing
traffic from a single
TCP connection. A typical client application, however, opens
several TCP connections,  many of
which are accompanied by DNS requests and
responses. This additional traffic presents more opportunities for
correlation attacks.
%
This paper quantifies how DNS traffic can make Tor users more vulnerable
to correlation attacks.
We investigate how incorporating DNS traffic can make existing correlation attacks more powerful
and how DNS lookups can leak information to third parties about anonymous
communication. We \first develop a
method to identify the DNS resolvers of Tor exit relays; \second develop
a new set of correlation attacks (\name attacks) that incorporate DNS traffic to improve
precision; \third analyze the Internet-scale effects of these new attacks on Tor
users; and \fourth develop improved methods to evaluate correlation
attacks.
%
First, we find that there exist adversaries who can mount \name
attacks: for example, 
Google's DNS resolver observes almost 40\% of all DNS 
requests exiting the Tor network. We also find that DNS requests also often traverse ASes
that the corresponding TCP connections do not transit, enabling
additional ASes to gain information about Tor users' traffic.
%%% 
We then show that an adversary who can mount a \name attack can often
determine the website that a Tor user is visiting with perfect
precision, particularly for less popular websites where the set of DNS
names associated with that website may be unique to the site.
We also use the
Tor Path Simulator (TorPS) in combination with traceroute data from
vantage points co-located with Tor exit relays to estimate the power of
AS-level adversaries who might mount \name attacks in practice. 
\end{abstract}
