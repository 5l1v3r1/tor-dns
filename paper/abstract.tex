\begin{abstract}
Many research projects investigated the practicality of end-to-end correlation
attacks on the Tor network.  All work, however, only focused on the TCP stream
between clients and servers.  The client's accompanying DNS requests were
ignored, and DNS's impact on the Tor network's anonymity is still poorly
understood.
%
In this work, we explore the impact of DNS on Tor users' anonymity.
Specifically, we investigate if DNS can improve existing attacks, and how much
information it leaks to third parties.  To this end, we \first develop a novel
method to identify the DNS resolvers of Tor exit relays, we \second show how
existing website fingerprinting (WF) attacks can be improved by observing DNS
requests, we \third analyze the Internet-scale impact of our attack on Tor
users, and we \fourth present an improved method to evaluate correlation
attacks.
%
Our results show that Google's DNS resolver observes almost 40\% of all DNS
requests exiting the Tor network. Furthermore, DNS requests often transit ASs
that subsequent TCP streams do not transit, enabling additional organizations to
snoop on Tor traffic.
By correlating the output of WF
attacks with observed DNS requests, WF+DNS attacks are perfectly precise for
unpopular websites, further motivating the introduction of WF defenses in Tor.
\fixme{Add insights about TorPS simulations.}
Finally, some of our findings generalize to other anonymity networks.
\end{abstract}
