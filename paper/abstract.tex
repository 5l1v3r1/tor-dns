\begin{abstract}
Previous correlation attacks on the Tor network have focused solely on
the attacks that are possible from analyzing a single TCP connection between
a client and a server. In reality, of course, a client application opens
several TCP connections to handle a single application request; many of
those TCP connections are also accompanied by DNS requests and
responses. This additional traffic presents many more opportunities for
an adversary to launch a correlation attack.
%
This paper quantifies the effects of DNS traffic on Tor users' anonymity.
We investigate how DNS can make existing attacks more powerful, as well
as how DNS lookups can leak information to third parties about anonymous
communication. We \first develop a
method to identify the DNS resolvers of Tor exit relays; \second show how
existing website fingerprinting (WF) attacks are stronger when they incorporate DNS
traffic; \third analyze the Internet-scale effects of these new attacks on Tor
users; and \fourth present an improved method to evaluate correlation
attacks.
%
We find that Google's DNS resolver observes almost 40\% of all DNS
requests exiting the Tor network. DNS requests also often traverse ASes
that the corresponding TCP connections do not transit, enabling
additional ASes to gain information about Tor users' traffic.
%%% 
\xxx{Abstract is very unclear from here down.}
By correlating the output of WF
attacks with observed DNS requests, WF+DNS attacks are perfectly precise for
unpopular websites, further motivating the introduction of WF defenses in Tor.
\fixme{Add insights about TorPS simulations.}
Finally, some of our findings generalize to other anonymity networks.
\end{abstract}
