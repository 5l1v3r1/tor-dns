\section{Discussion}
\label{sec:discussion}

In this section, we briefly discuss the ethics of our research and
possible ways to defend against \name attacks.

\subsection{Ethical considerations}
\label{sec:ethics}

Section~\ref{sec:load-freq} discusses how we set up an exit relay to
determine the number of DNS requests per five minute interval.  Since
our exit relay was forwarding traffic of Tor users, we contacted our
university's institutional review board (IRB) before running the
experiment.  Our IRB deemed that this research did not fall within the
realm of human subjects research.  In addition to contacting our IRB, we
adhered to The Tor Project's ethics guidelines~\cite{ethics-guidelines}.
Specifically, \first we ensured that we only collected data that is safe
to publish, \second we only collected data we needed, and \third we
limited the granularity of the data to minimize the likelihood of
reidentification.  The risk to Tor users of this experiment negligible.
As for the benefits, by conducting this experiment, we can improve our
understanding of the risks that DNS poses to the anonymity of Tor users
and use this understanding to improve protection for Tor users in the
future.
Thus, we believe that the benefits of our experiment outweigh the risks.

\subsection{Defending against \name attacks}

We now discuss ways to defend against \name attacks.  We distinguish
between short-term solutions that can be implemented quickly
(Section~\ref{sec:short-term}), and long-term solutions that need significantly
more work (Section~\ref{sec:long-term}).

\subsubsection{Short-term solutions}
\label{sec:short-term}

Operators of exit relays face a dilemma: they must either operate their own
resolver, which exposes DNS queries to network adversaries; or, they
must use a third-party DNS resolver, which exposes DNS queries to a
third party.  Clearly, the goal is to minimize exposure of DNS requests,
but there are several dimensions to this.  In lieu of substantial DNS
protocol improvements, we can envision three extreme design points,
in which \emph{all} exit relays use \first Google's DNS resolver;
\second their own, local resolver; or \third the resolver provided by
their ISP.  Table~\ref{tab:setup-comparison}
summarizes the important tradeoffs for these three setups; the rest of
this section discusses these design points in more detail.

If all exit relays were to use Google's DNS resolver, the company would
obtain metadata about the activity of all Tor users, which runs counter to
Tor's design goal of distributing trust.  We clearly should avoid this
scenario. Unfortunately, some of Tor's pluggable transports
have {\em already} made this design choice and should likely be updated
to mitigate user risk.  For example, Fifield \ea's~\cite{Fifield2015a}
Meek used Google's cloud infrastructure
to reach the Tor network up until May 2016.
% died May 13, 2006. See https://lists.torproject.org/pipermail/tor-talk/2016-June/041699.html
Thousands of Meek clients
selected exit relays that use Google's DNS resolver, which means that Google
saw both traffic entering and, partially, exiting the Tor network.

Next, consider a Tor network that only uses local resolvers.  In this
case, Tor is fully independent of third-party resolvers, at the cost of
each iterative DNS query being exposed to a diverse set of ASes in the
network, allowing several parties to learn the DNS queries of Tor users.

Finally, all exit
relays could simply use their ISP-provided resolver.  This would minimize the
network exposure of DNS requests as resolvers are frequently in the same AS as
exit relays, and AS-level adversaries would be unable to distinguish
between DNS requests from exit relays and unrelated ISP customers.  This
setup introduces the possibility of misconfigured and censored DNS
resolvers~\cite[\S~4.1]{Winter2014b}.  Besides, just a few ASes---OVH, for
example---host a disproportionate amount of exit relays, turning them into the
centralized data sinks that Tor aims to avoid.

Considering the above, we believe that exit relay operators should avoid
public resolvers such as Google and OpenDNS.  Instead, they should
either use the resolvers provided by their ISP, or run their own,
particularly of the operator's
ISP already hosts many other exit relays.  Local
resolvers can further be optimized to minimize information leakage,
by (for example) enabling QNAME minimization~\cite{qname-minimization}.

\begin{table}[t]
  \renewcommand{\tabcaptext}{A comparison between three design points for DNS
          resolver configuration, assuming all Tor exit relays use the
          setup in question.  Solid black circles are most desirable.}
	\topcap{\tabcaptext}
	\centering
	\begin{tabular}{l c c c}
	\toprule
	\textbf{Setup} &
	\begin{tabular}{@{}c@{}}\textbf{Network-level}\\\textbf{Protection}\end{tabular} &
	\begin{tabular}{@{}c@{}}\textbf{Avoiding}\\\textbf{Centralization}\end{tabular} &
	\begin{tabular}{@{}c@{}}\textbf{Response}\\\textbf{Quality}\end{tabular} \\
	\midrule
	All Google & \RIGHTcircle & \Circle & \CIRCLE \\
	All Local & \Circle & \CIRCLE & \CIRCLE\\
	All ISP & \CIRCLE & \RIGHTcircle & \RIGHTcircle \\
	\bottomrule
	\end{tabular}
	\bottomcap{\tabcaptext}
       \label{tab:setup-comparison}
\end{table}

In addition to making recommendations to exit relay operators, we can
remotely influence the cache of each exit relay's resolver.  For
example, using {\tt exitmap}, we can resolve potentially sensitive DNS
domain names over each exit relay continuously, right before its TTL is
about to expire.  In such a setup, an attacker gains no advantage from
observing DNS traffic from the exit relays is always in every exit
relay's DNS resolver cache.  This approach may not scale, considering
the potentially large number of domain names that would need to be
cached (recall that the long tail of unpopular sites are most vulnerable
to \name attacks), but it allows us to eliminate DNS-based correlation
attacks for a select number of sites.

Finally, Tor can fix the Tor clipping bug and consider significantly increasing the
minimum TTL for the DNS cache at exits to make \name attacks less precise.
This adjustment requires finding the longest acceptable TTL that does not have a
notable negative detriment to user experience.

\subsubsection{Long-term solutions}
\label{sec:long-term}

Additional practical defenses are on the horizon.  Zhu \ea~\cite{Zhu2015a} proposed
T-DNS, which employs several TCP optimizations to transport the DNS protocol
over TLS and TCP.  Since T-DNS is based on TCP, Tor can transport these queries.
The TLS layer provides confidentiality between exit relays and their resolvers.
Finally, site operators whose users are particularly concerned about
safety should offer an onion service as an alternative.  Facebook, for example,
set up \url{facebookcorewwwi.onion}.  When
connecting to the onion service, Tor users never leave the Tor network, and
hence no not need DNS.

Deploying defenses against website fingerprinting attacks in Tor should be an
important long-term goal, as well.
Although growing the Tor network will help defend against \name attacks to some
degree, the most important change is to
deploy defenses against these attacks in the first place.  Since \name attacks
significantly increase precision of website fingerprinting attacks, defenses
should be designed to significantly reduce the recall of website fingerprinting
attacks, even when the website fingerprinting attack is configured to sacrifice
precision for recall.
