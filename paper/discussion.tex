\section{Discussion}
\label{sec:discussion}

\subsection{Mitigation}
\begin{itemize}
	\item What do we recommend to exit relay operators to mitigate the issue?
	\item Is it better to run your own, local resolver?
		\begin{itemize}
			\item Could be dangerous since some resolvers write stuff to disk:
		\url{https://github.com/NullHypothesis/exitmap/commit/fa1cd0ba6ce3389d3d6fe1dfc7144e2747320944#commitcomment-16886249}
		\end{itemize}
	\item Or should you use your provider's resolver?
	\item \ldots or something else?
	\item Would qname minimisation help?
	\item Ideally, use only onion services, or perhaps OnionNS, or the GNU Name
		System~\cite{Wachs2014a}.
	\item Can exit relays use private information retrieval to fetch domain
		names anonymously?
	\item Is there a trade-off between enforced min DNS TTL and the impact of
		DNS poisoning?
	\item A tool that forces Tor-exits to regularly resolve uniquely identifying
		domain names?
	\item T-DNS~\cite{Zhu2015a}
\end{itemize}

\subsection{Short-term solutions}
Exit relays seem stuck in a predicament, having to choose between running their
own resolver (which exposes DNS queries to network adversaries) or using a
third-party resolver (which exposes DNS queries to a third party).

It appears tempting to have exit relays use third-party resolvers such as
Google's 8.8.8.8.  This resolver has three key advantages; (\emph{i}) it does
not filter DNS requests; (\emph{ii}) it is fast and well-maintained; and
(\emph{iii}) it is anycast using world-wide data centers, which minimizes
exposure to network-level adversaries.

Indeed, figure~\ref{fig:exit-resolvers} shows that Google occasionally gets to see more
than 40\% of all DNS requests exiting the Tor network.
Note that one variant of Fifield et al.'s~\cite{Fifield2015a} censorship
circumvention system meek leverages Google's cloud infrastructure to reach the
Tor network.  Inevitably, many of the thousands of meek clients will select
exit relays that use Google's DNS resolver, which means that Google gets to see
both traffic entering and, partially, exiting the Tor network.


\subsection{Long-term solutions}

Practical defenses are on the horizon.  Zhu et al.~\cite{Zhu2015a} proposed
T-DNS, which transports the DNS protocol over TLS and TCP.  Since the protocol
uses TCP, it can be sent over Tor without any modification.  The TLS layer
guarantees confidentiality between Tor clients and their resolvers.

Meanwhile, Google is experimenting with DNS over HTTPS.


Better fingerprinting defenses.

More onion services.
