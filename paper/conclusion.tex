\section{Conclusion}
\label{sec:conclusion}

In this paper, we have demonstrated how AS-level adversaries can use DNS
traffic from Tor exit relays to launch more effective correlation
attacks, which link the sender and receiver of traffic in the Tor
network.  We show how an attacker can use DNS query traffic to mount
perfectly precise website fingerprinting attacks: Due to the fact that
many websites have at least one DNS domain that is unique to that
website, an attacker can often identify the website that a client is
visiting with perfect accuracy, given the ability to observe the
client's set of DNS requests.

Given this more powerful fingerprinting method, we showed that the
threat of \name attacks against the Tor network is clear and present. We
first developed a method to identify the DNS resolver for each Tor exit relay,
and found that a set of exit relays comprising 40\% of all Tor exit
relay bandwidth use the Google public DNS servers. Although this
concentration of DNS query traffic reduces the expanse of ASes that can
see DNS query traffic emanating from exit nodes, this configuration
nonetheless gives a single administrative entity considerable visibility
into the traffic that is exiting the Tor network. The Tor Project should
take steps to ensure that the network maintains more diversity into how
exit relays resolve DNS domains.

On the other hand, iterative DNS resolution from a Tor exit relay allows
a greater number of ASes to observe DNS queries from Tor exit relays.
For the Alexa most popular 1,000 websites, about 60\% of the DNS lookups
required to resolve the website's DNS domain name traversed ASes that
were {\em not} on the end-to-end network path between the client and the
resolved IP address for the webserver.  To mitigate the risk of
correlation attacks in light of this finding, we suggest that local DNS
resolvers on Tor exit relays implement privacy-preserving techniques
such as DNS QNAME minimization, which minimizes the amount of
information about the domain name that each iterative query contains.

Website fingerprinting attacks have long been a concern for the Tor
network. The attacks that we present in this paper show that, when
incorporating DNS query traffic, these attacks become even more accurate
and powerful. We hope these findings underscore the urgency of
eventually deploying strong defenses against fingerprinting attacks on
the Tor network. 
