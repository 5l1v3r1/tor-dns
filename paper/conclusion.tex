\section{Conclusion}
\label{sec:conclusion}

In this paper, we have demonstrated how AS-level adversaries can use DNS traffic
from Tor exit relays to launch more effective website fingerprinting attacks, to
learn what websites Tor users are visiting.  Mapping DNS traffic to websites is
highly accurate even with simple techniques, and improves the precision when
monitoring relatively unpopular websites.  We further developed a method to
identify the DNS resolver for each Tor exit relay, and found that a set of exit
relays comprising 40\% of all Tor exit relay bandwidth use the Google public DNS
servers.  Although this concentration of DNS query traffic reduces the expanse
of ASes that can see DNS query traffic emanating from exit nodes, this
configuration nonetheless gives a single administrative entity considerable
visibility into the traffic that is exiting the Tor network. Tor relay operators
should take steps to ensure that the network maintains more diversity into how
exit relays resolve DNS domains.  To mitigate the risk of website fingerprinting
attacks in light of our work, we suggest that local DNS resolvers on Tor exit
relays implement privacy-preserving techniques such as DNS QNAME minimization,
which minimizes the amount of information about the domain name that each
iterative query contains.  We publish all our code, data, and replication
instructions on our project page, which is available online at
\url{https://nymity.ch/tor-dns/}.
