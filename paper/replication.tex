\documentclass{article}

\usepackage[T1]{fontenc}
\usepackage[backend=biber,backref=true,maxbibnames=100]{biblatex}
\usepackage[scaled=0.8]{roboto} % For a pretty sans-serif font.
\usepackage[scaled=0.8]{beramono} % For a pretty monospaced font.
\usepackage{mathptmx} % Use times for serif font.
\usepackage{xcolor} % For colourful links.
\usepackage{listings} % For source code listings.
\usepackage{xspace} % For reasonable spacing in custom commands.
\usepackage{hyperref} % For clickable links.
\usepackage{upquote} % No curly quotes in listings.
\usepackage{a4wide}

\bibliography{references}

\urlstyle{sf} % Nice-looking URLs.

\definecolor{darkblue}{rgb}{0,0,0.5}
\definecolor{lightgray}{rgb}{0.93,0.93,0.93}

\lstset{
	backgroundcolor=\color{lightgray},
	basicstyle=\small\ttfamily,
	keepspaces=true,
	breakatwhitespace=true,
	breaklines=true,
	numbers=left,
	numbersep=3pt,
	numberstyle=\scriptsize\color{gray},
}

\hypersetup{
	pdftitle={Instructions on replicating our experiments},
	pdfauthor={Benjamin Greschbach, Tobias Pulls, Laura M. Roberts, Philipp Winter, Nick Feamster},
	colorlinks=true,
	urlcolor=darkblue,
	linkcolor=darkblue,
	citecolor=darkblue
}

\newcommand{\name}{DefecTor\xspace}
\newcommand{\first}{(\emph{i})\xspace}
\newcommand{\second}{(\emph{ii})\xspace}
\newcommand{\third}{(\emph{iii})\xspace}
\newcommand{\fourth}{(\emph{iv})\xspace}
\newcommand{\fifth}{(\emph{v})\xspace}

\title{Instructions on replicating our experiments}

\author{%
	Benjamin Greschbach --- \href{mailto:bgre@kth.se}{\nolinkurl{bgre@kth.se}} \\
	Tobias Pulls --- \href{mailto:tobias.pulls@kau.se}{\nolinkurl{tobias.pulls@kau.se}} \\
	Laura M. Roberts --- \href{mailto:laurar@princeton.edu}{\nolinkurl{laurar@princeton.edu}} \\
	Philipp Winter --- \href{mailto:pwinter@cs.princeton.edu}{\nolinkurl{pwinter@cs.princeton.edu}} \\
	Nick Feamster --- \href{mailto:feamster@cs.princeton.edu}{\nolinkurl{feamster@cs.princeton.edu}}
}

\begin{document}

\maketitle

In this document, we provide instructions on how to replicate the results from
our research paper ``The Impact of DNS on Tor's Anonymity''.\footnote{Available
online at \url{https://nymity.ch/tor-dns/dns-tor.pdf}.} Each section discusses
the replication of a specific experiment, providing both code and data necessary
to replicate.  Our project page is available online at
\url{https://nymity.ch/tor-dns/}.

\tableofcontents

\newpage

\section{Replicating AS exposure of DNS requests}
Here is how you can replicate the results of Subsection~IV.A.  First, obtain the
Python 2 tool ddptr that we have developed for this measurement task:

\begin{lstlisting}
git clone https://github.com/NullHypothesis/ddptr.git
\end{lstlisting}

Next, you need two files to continue; \first a list of websites such as the
Alexa top 1,000~\cite{alexatop1k}, which you should save as \path{top-1k.csv};
\second a file \path{ipasn.dat}, which enables looking up AS numbers for given IP
addresses.  The library pyasn documents how you can generate this
file~\cite{pyasn}.  After having fetched \path{top-1k.csv}, \path{ipasn.dat} and
\path{ddptr.py}, you are ready to run it as follows:

\begin{lstlisting}
sudo ./ddptr.py --fqdn-file top-1k.csv ipasn.dat 2>&1 | tee top-1k-ddptr.log
\end{lstlisting}

The resulting log files contains plenty of information, but only the lines
``Exposure is \ldots'' are necessary to create Figure~2.  You can extract all
exposure values from the log files by running:

\begin{lstlisting}
echo exposure > exposures.csv
grep Exposure top-1k-ddptr.log | awk '{print $NF}' >> exposures.csv
\end{lstlisting}

The file \path{exposures.csv} then contains all exposure values, one per line,
including a header line.  It is easy to take the file and plot it with a tool of
your choice.

\section{Replicating DNS resolver mapping}
Here is how you can replicate the results of Subsection~IV.B.  First, you need
to set up an authoritative DNS server for a domain that is under your control.
We set up a bind9 instance that was responsible for the zone *.tor.nymity.ch.
We configured bind9 to return the same IP address for any request for
*.tor.nymity.ch.  Next, you need the Python 2 exitmap scanner~\cite{exitmap}:

\begin{lstlisting}
git clone https://github.com/NullHypothesis/exitmap.git
\end{lstlisting}

Exitmap is a module-based scanning framework.  We developed a new module for
exitmap that is part of the following git repository.  You can clone it, and
then copy the module to exitmap's modules directory as follows:

\begin{lstlisting}
git clone https://github.com/NullHypothesis/tor-dns-tools.git
cp tor-dns-tools/exit-resolvers/dnsenum.py /path/to/exitmap/src/modules/
\end{lstlisting}

You are now ready to start the experiment.  On the machine running bind9, you
need to run tcpdump, to capture all DNS requests going to bind9:

\begin{lstlisting}
tcpdump -n "udp and port 53" -w dns-requests.pcap
\end{lstlisting}

Bind9 is now ready to serve DNS requests and tcpdump is capturing them to a
file.  You can now run exitmap to generate the data:

\begin{lstlisting}
exitmap --build-delay 0.5 dnsenum
\end{lstlisting}

There are various flags for exitmap that can improve the reliability of your
experiment.  You might be particularly interested in {\tt ---first-hop}.

We set up a cronjob to run exitmap automatically, every six hours.  First, edit
your crontab:
\begin{lstlisting}
crontab -u "$USER" -e
\end{lstlisting}
Then, add the following line to the crontab file:
\begin{lstlisting}
0 */6 * * * /path/to/exitmap -f /path/to/exitmaprc dnsenum
\end{lstlisting}

The file \path{exitmaprc} had the following content.  The key {\tt first\_hop}
is not necessary, but it makes scanning more reliable.  We set it to the
fingerprint of a Tor relay under our control.
\begin{lstlisting}
[Defaults]
first_hop = 9B94CD0B7B8057EAF21BA7F023B7A1C8CA9CE645
build_delay = 0.5
analysis_dir = /path/to/analysis_directory/
tor_dir = /path/to/tor_directory/
\end{lstlisting}

Having set up cron, exitmap will now run periodically, and the data you need
will be captured in the pcap file that is created on the machine running bind9.
Once you are ready to analyze the data, the \path{README} file of our GitHub
repository\footnote{Available online at
\url{https://github.com/NullHypothesis/tor-dns-tools/tree/master/exit-resolvers}.}
discusses how you can turn the pcap file into visualizations:

\section{Replicating number of DNS requests on exit relays}
Because of the ethical issues of recording data on exit relays, we developed a
patch for tshark that allows us to reduce the granularity of timestamps.  That
way, we are able to record the number of DNS requests in a given time period
without payload or further information.

The following listing contains that patch.  It only modifies two lines of
tshark's code.  Save it to the file {\tt tshark-binned-timestamps.patch}, and
then obtain the source code of wireshark in version 1.12.1, and place it in the
same directory as the patch file.

\begin{lstlisting}
diff -Naur wireshark-1.12.1.orig/epan/frame_data.c wireshark-1.12.1/epan/frame_data.c
--- wireshark-1.12.1.orig/epan/frame_data.c     2014-09-16 18:09:03.000000000 +0200
+++ wireshark-1.12.1/epan/frame_data.c  2016-07-21 19:31:56.392000000 +0200
@@ -310,8 +310,8 @@
   fdata->flags.has_phdr_comment = (phdr->opt_comment != NULL);
   fdata->flags.has_user_comment = 0;
   fdata->color_filter = NULL;
-  fdata->abs_ts.secs = phdr->ts.secs;
-  fdata->abs_ts.nsecs = phdr->ts.nsecs;
+  fdata->abs_ts.secs = phdr->ts.secs - (phdr->ts.secs % 3600);
+  fdata->abs_ts.nsecs = 0;
   fdata->shift_offset.secs = 0;
   fdata->shift_offset.nsecs = 0;
   fdata->frame_ref_num = 0;
\end{lstlisting}

Finally, you can apply the patch as follows.

\begin{lstlisting}
patch -p0 < tshark-binned-timestamps.patch
\end{lstlisting}

After compiling tshark (which is part of wireshark), you can now run our
modified version as follows.  The command does not resolve IP addresses; only
captures UDP traffic on port 53; only prints timestamps, and no payload; and
only prints DNS requests, and no responses.  The output of the command is coarse
timestamps, one per line, of when tshark captured a DNS request.

\begin{lstlisting}
sudo ./tshark -n -f 'udp port 53' -T fields -e frame.time_epoch -Y 'dns.qry.type == 1 and dns.flags.response == 0'
\end{lstlisting}

\section{Replicating DNS requests stats}
This is how you replicate the DNS-related statistics from Section~V.B.  First
download and extract the Alexa top one million websites dataset with extracted
DNS requests:

\begin{lstlisting}
wget FIXME/alexa1mx5-extracted.tar.gz
tar -zxf alexa1mx5-extracted.tar.gz
\end{lstlisting}

Get the Alexa file we used when gathering the data~\cite{alexatop1k} and the
CloudFlare IPv4 addresses\footnote{Available online at
\url{https://www.cloudflare.com/ips-v4}.} at the time of gathering:

\begin{lstlisting}
wget FIXME/top-1m.csv
wget FIXME/ips-v4
\end{lstlisting}

Get and install our dnsstats tool using Go\footnote{Available online at
\url{https://golang.org/}.}:

\begin{lstlisting}
go get github.com/pylls/fingerprintor/cmd/dnsstats
\end{lstlisting}

Finally, run dnsstats on the downloaded data:

\begin{lstlisting}
dnsstats alexa1mx5-extracted/
\end{lstlisting}

This produces a large number of statistics, including those used in Section~V.B.

\section{Replicating DNS to website classification}
This is how you replicate the DNS to website classification from Section~V.B.
First download and extract the Alexa top one million websites dataset with
extracted DNS requests:

\begin{lstlisting}
wget FIXME/alexa1mx5-extracted.tar.gz
tar -zxf alexa1mx5-extracted.tar.gz
\end{lstlisting}

Get and install our dns2site tool using Go:

\begin{lstlisting}
go get github.com/pylls/fingerprintor/cmd/dns2site
\end{lstlisting}

For the closed world metrics, run:

\begin{lstlisting}
dns2site -sites 1000000 -instances 5 -open 0
\end{lstlisting}

For the open world metrics, run:

\begin{lstlisting}
dns2site -sites 500000 -instances 5
\end{lstlisting}

Without the \texttt{-open} parameter, the dns2site tool determines a reasonable
open world size based on our conservative power-law distribution,
which should be around 433,000.

\section{Replicating \name attacks}
This is how you replicate the \name results figures from Section~VI.  First
download and extract the traffic traces from our 1,000x100+100,000 WF dataset:

\begin{lstlisting}
wget FIXME/alexa1kx100+100k-extracted.tar.gz
tar -zxf alexa1kx100+100k-extracted.tar.gz
\end{lstlisting}

Get and install our fpt tool using Go:

\begin{lstlisting}
go get github.com/pylls/fingerprintor/cmd/fpt
\end{lstlisting}

To generate data for Figure~8, run:

\begin{lstlisting}
fpt -sites 1000 -instances 100 -open 100000 -pmin 0 -pmax 100 -pstep 5 -alexa 10000
\end{lstlisting}

The result is written to stdout and three files: one CSV file for precision,
one CSV file for recall, and a log of all output. All filenames capture relevant
parameters and are only created upon experiment completion.

To generate data for Figure~9, run as a shellscript:

\begin{lstlisting}
for i in 1 10 100 1000 10000 100000 1000000 10000000 100000000
do
  fpt -sites 1000 -instances 100 -open 100000 -pmin 100 -alexa $i
done
\end{lstlisting}

Manual assembly of the CSV files is needed. For Figures~10a,~10b,~10c, and~10d,
run as a shellscript:

\begin{lstlisting}
# rounds
for i in 0 300 600 900 1200 1500 1800 2100 2400 2700 3000
do
  fpt -sites 1000 -instances 100 -open 100000 -pmin 33 -pmax 33 -alexa 10000 -r $i
done
# window size
for i in 90 180 360 540 720 900 1080 1260 1440 1620 1800
do
  for j in 10000 100000
  do
  fpt -sites 1000 -instances 100 -open 100000 -pmin 33 -pmax 33 -window $i -alexa $j
  done
done
# Tor network scale
for i in 1.0 2.0 3.0 4.0 5.0 6.0 7.0 8.0 9.0 10.0
do
  for j in 10000 100000
  do
  fpt -sites 1000 -instances 100 -open 100000 -pmin 33 -pmax 33 -scaletor $i -alexa $j
  done
done
# different distributions
for i in conpl realpl conuni realuni
do
  for j in 0 10 100 1000 10000 100000 1000000 10000000 100000000
  do
    fpt -sites 1000 -instances 100 -open 100000 -pmin 33 -pmax 33 -simdist $i -alexa $j
  done
done
\end{lstlisting}

Manual assembly of the CSV files is needed.
Generating the figures takes \emph{days} on modern
hardware: we spent well over a CPU core-year in total on the figures.

\section{Replicating the DNS Alexa top 1,000,000 data\-set}
This is how you replicate the steps we used to gather the Alexa top 1,000,000
dataset used in Section~V.B.  First, get and install three tools using Go:

\begin{lstlisting}
go get github.com/pylls/fingerprintor/cmd/{server,tbbdnsw}
go get github.com/pylls/fingerprintor/cmd/extractdns
\end{lstlisting}

We use a worker-server model, where a server instructs workers to browse to a
site and return the observed DNS requests. Download an Alexa file with top
sites~\cite{alexatop1k} and run the server:

\begin{lstlisting}
server -f data -s 5 -t 30 top-1m.csv
\end{lstlisting}
The server will instruct workers to collect in total
five samples of the sites in \path{top-1m.csv}, using 30 seconds per site visit,
and store the results in the data folder. By default, the server listens on
port 55555 on all interfaces.

Download a fresh copy of Tor Browser from \url{https://torproject.org} and extract it.
Open \path{Browser/TorBrowser/Data/Browser/profile.default/preferences/}
and put the
following \emph{at the bottom} of\\ \path{extension-overrides.js}:

\begin{lstlisting}
user_pref("app.update.enabled", false);
user_pref("extensions.torlauncher.prompt_at_startup", false);
user_pref("extensions.torlauncher.start_tor", false);
user_pref("datareporting.healthreport.nextDataSubmissionTime", "1559373924100");
user_pref("datareporting.policy.firstRunTime", "1559287524100");
user_pref("extensions.torbutton.lastUpdateCheck", "1559287542.7");
user_pref("extensions.torbutton.show_slider_notification", false);
user_pref("extensions.torbutton.updateNeeded", false);
user_pref("extensions.torbutton.versioncheck_url", "");
user_pref("extensions.torbutton.versioncheck_enabled", false);
user_pref("network.proxy.proxy_over_tls", false);
user_pref("network.proxy.socks", "");
user_pref("network.proxy.socks_port", 0);
user_pref("network.proxy.socks_remote_dns", false);
\end{lstlisting}

Download the latest release of dumb-init from
\url{https://github.com/Yelp/dumb-init/releases}. We need a minimal init system
to clean up the many processes we will be creating in Docker.
Copy the following into a
new file named \texttt{Dockerfile}\footnote{Based on
\url{https://github.com/jfrazelle/dockerfiles/tree/master/tor-browser}.}:

\begin{lstlisting}
FROM debian:jessie
MAINTAINER Tobias Pulls <tobias.pulls@kau.se>

RUN apt-get update && apt-get install -y \
	xvfb \
	libpcap-dev \
	libasound2 \
	libdbus-glib-1-2 \
	libgtk2.0-0 \
	libxrender1 \
	libxt6 \
	xz-utils \
  xauth \
  psmisc \
	--no-install-recommends

COPY dumb-init*_amd64.deb /
RUN dpkg -i dumb-init*.deb
RUN rm dumb-init*.deb && apt-get clean && rm -rf /var/lib/apt/lists/* /tmp/* /var/tmp/*

ENV HOME /home/user
ENV LANG C.UTF-8

# create user (start-tor-browser.sh prevents us from running as root)
RUN useradd --create-home --home-dir $HOME user

COPY tbbdnsw $HOME/
COPY tor-browser_en-US $HOME/tor-browser_en-US

RUN chown -R user:user $HOME \
	&& chmod +x $HOME/tbbdnsw \
  && setcap 'CAP_NET_RAW+eip CAP_NET_ADMIN+eip' $HOME/tbbdnsw

WORKDIR $HOME
USER user
ENTRYPOINT ["dumb-init", "--"]
\end{lstlisting}

Build the docker container and start a worker:

\begin{lstlisting}
docker build -t pulls/worker  .
docker run --privileged -d pulls/worker ./tbbdnsw <IP:port>
\end{lstlisting}
The worker will repeatedly try to connect to the server at {\tt <IP:port>}. We
successfully ran 120 workers each 1U blades with Intel(R) Xeon(R) CPU E5-2650 0
@ 2.00GHz with 62GB RAM. The server was hosted on a laptop with SSD storage.

Finally, to extract the DNS data from the resulting pcaps use the
extractdns tool:

\begin{lstlisting}
extractdns -o results/ data/
\end{lstlisting}
Where data is the folder the server stored the data in and results is the folder
to store the extracted data in.

\section{Replicating the WF dataset}
To replicate our WF dataset used in Section~VI, get and install the following
four tools using Go:
% server
\begin{lstlisting}
go get github.com/pylls/fingerprintor/cmd/{server,tbbw}
go get github.com/pylls/fingerprintor/cmd/{torlogext,fext}
\end{lstlisting}

Download an Alexa file with top sites~\cite{alexatop1k} and run:

\begin{lstlisting}
server -f data -s 5 -t 60 -o .torlog top-1m.csv
\end{lstlisting}
The server will instruct workers to collect in total
five samples of the sites in \path{top-1m.csv}, using 60 seconds per site visit,
and store the results in the data folder with the suffix \texttt{.torlog}.
By default, the server listens on port 55555 on all interfaces.

% configure Tor
Download a fresh copy of Tor Browser from \url{https://torproject.org} and
extract it.  Open
\path{Browser/TorBrowser/Data/Browser/profile.default/preferences/} and put the
following \emph{at the bottom} of\\ \path{extension-overrides.js}:

\begin{lstlisting}
user_pref("app.update.enabled", false);
user_pref("extensions.torlauncher.prompt_at_startup", false);
user_pref("datareporting.healthreport.nextDataSubmissionTime", "1759373924100");
user_pref("datareporting.policy.firstRunTime", "1759287524100");
user_pref("extensions.torbutton.lastUpdateCheck", "1759287542.7");
user_pref("extensions.torbutton.show_slider_notification", false);
user_pref("extensions.torbutton.updateNeeded", false);
user_pref("extensions.torbutton.versioncheck_url", "");
user_pref("extensions.torbutton.versioncheck_enabled", false);
\end{lstlisting}

Open {\tt Browser/TorBrowser/Data/Tor/torrc} and add:

\begin{lstlisting}
LogTimeGranularity 1
UseEntryGuards 0
\end{lstlisting}

% build custom tor
Next, we need to build a custom tor binary for Tor Browser that logs all
incoming and outgoing cells using Tor's logging framework. First, get the tor
source code:

\begin{lstlisting}
git clone https://git.torproject.org/tor.git
\end{lstlisting}
Follow the instructions in the {\tt INSTALL} file. Once you can build tor,
apply the below patch to {\tt src/or/relay.c} and run make:
% clients
\begin{lstlisting}
diff -Naur tor/src/or/relay.c relay.c
--- tor/src/or/relay.c	2016-08-02 18:04:05.796809070 +0200
+++ relay.c	2016-08-02 18:03:50.036572253 +0200
@@ -585,6 +585,10 @@
     return -1;
   }

+  log_notice(LD_GENERAL, "OUTGOING CIRC %u STREAM %d COMMAND %s(%d) length %zu",
+             circ->n_circ_id, stream_id, relay_command_to_string(relay_command),
+             relay_command, payload_len);
+
   memset(&rh, 0, sizeof(rh));
   rh.command = relay_command;
   rh.stream_id = stream_id;
@@ -1453,6 +1457,9 @@
         ;
     }
   }
+  log_notice(LD_GENERAL, "INCOMING CIRC %u STREAM %d COMMAND %s(%d) length %d",
+             circ->n_circ_id, rh.stream_id,
+             relay_command_to_string(rh.command), rh.command, rh.length);

   /* either conn is NULL, in which case we've got a control cell, or else
    * conn points to the recognized stream. */
\end{lstlisting}
Copy {\tt src/or/tor} to {\tt Browser/TorBrowser/tor}.

Download the latest release of dumb-init from
\url{https://github.com/Yelp/dumb-init/releases}.
Copy the following into a
new file named \texttt{Dockerfile}:

\begin{lstlisting}
FROM debian:jessie
MAINTAINER Tobias Pulls <tobias.pulls@kau.se>

RUN apt-get update && apt-get install -y \
	xvfb \
	libpcap-dev \
	libasound2 \
	libdbus-glib-1-2 \
	libgtk2.0-0 \
	libxrender1 \
	libxt6 \
	xz-utils \
  xauth \
  psmisc \
	--no-install-recommends

COPY dumb-init*_amd64.deb /
RUN dpkg -i dumb-init*.deb
RUN rm dumb-init*.deb && apt-get clean && rm -rf /var/lib/apt/lists/* /tmp/* /var/tmp/*

ENV HOME /home/user
ENV LANG C.UTF-8

# create user (start-tor-browser.sh prevents us from running as root)
RUN useradd --create-home --home-dir $HOME user

COPY tbbw $HOME/
COPY tor-browser_en-US $HOME/tor-browser_en-US

RUN chown -R user:user $HOME \
	&& chmod +x $HOME/tbbw \
  && setcap 'CAP_NET_RAW+eip CAP_NET_ADMIN+eip' $HOME/tbbdnsw

WORKDIR $HOME
USER user
ENTRYPOINT ["dumb-init", "--"]
\end{lstlisting}

Build the docker container and start a worker:

\begin{lstlisting}
docker build -t pulls/worker  .
docker run --privileged -d pulls/worker ./tbbw <IP:port>
\end{lstlisting}

Finally, to extract the data from the resulting torlog-files use the
torlogext and fext tools:

\begin{lstlisting}
torlogext -o results/ data/
fext results
\end{lstlisting}
The torlogext file generates cell-files with the format used by
Wang et al.~\cite{Wang2014a}, and the fext tool extracts features for Wa-kNN.


%
%
%
%
%
%
\section{Replicating Our TorPS Simulations}
There are two main components you need in order to recreate the work of Section VII: 
\first~TorPS and \second~traceroutes between client ASes and Tor guards and between Tor 
exits and destinations. (The scripts we mention below have ``usage'' comments at the 
top of their files.)

\subsection{TorPS}
Grab the code for TorPS from the repository \url{https://github.com/torps/torps}.
Make sure you run TorPS on a computer 
that has enough memory. Otherwise, you will encounter an out-of-memory segfault (e.g., 4GB of 
memory is not enough). In order to run TorPS, go to their github page, and follow 
their instructions for processing Tor consesuses and descriptors. Here's the command 
we ran for that step:
\begin{lstlisting}
python pathsim.py process --start_year 2016 --start_month 3 --end_year 2016 --end_month 3 --in_dir in --out_dir out --initial_descriptor_dir in/server-descriptors-2013-02
\end{lstlisting}

After that you can run your desired simulations. Here's the command we used:
\begin{lstlisting}
python pathsim.py simulate --nsf_dir out/network-state-2016-03 --num_samples 100000 --trace_file here.pickle --user_model typical --format normal --num_guards 1 --guard_expiration 270 tor > torps_typical_my_final_model_1_guard_100000_samples.txt
\end{lstlisting}
In this command, a 100,000-sample simulation is run in which a client has ``typical'' 
behavior as given by our modified trace\textunderscore file (called {\tt ``here.pickle''}). 
The number of 
guards used is one, and guards expire sometime after 270 days. The file 
\path{torps_typical_my_final_model_1_guard_100000_samples.txt}
will contain your 
simulations for the 100,000 samples who visit our chosen websites. 

To make our modified trace file, we unpickled their included trace file 
(used {\tt unpickled.py} on \path{in/users2-processed.traces.pickle}) and 
modified it so that it would visit our desired websites.

Because we were interested in only DNS resolution, we used the trace file as an 
indication of when a client had visited a website and with what Tor circuit 
(or ``path'' as TorPS calls it), but we were not worried about any further 
website communications. 
See \path{final_model.py} to see how we have the 
client visit Facebook, Instagram, Google, Google Mail, Google Docs, 
Google Calendar, Startpage, Twitter, and Ixquick for our new trace 
file for the TorPS ``Typical'' model to use. 

\subsection{Traceroutes}
We use the RIPE Atlas platform to perform our traceroutes and use RIPE 
Atlas's Cousteau Python wrapper.\footnote{Available online at
\url{https://github.com/RIPE-NCC/ripe-atlas-cousteau}.}

\subsubsection{Traceroutes from client ASes to Tor Guards}
\emph{Source of traceroutes:} One RIPE Atlas probes in each of our five, chosen 
Tor client ASes.

\emph{Destination of traceroutes:} All Tor guards listed in the consensus 
files for March 2016.

The {\tt getGuardsAndExitsFromConsensusMonthDir.py} script parses consensuses 
and produces two text files: one that contains all of the guard IP addresses 
used in a particular month and one that contains all of the exit IP addresses 
used in that month.

The \path{do_udp_traceroutes.py} file in the directory
\path{clients_to_all_guards_forward}
will perform the traceroutes, and 
it takes as stdin the IP addresses you want to traceroute to. The script has 
our five client ASes hard-coded as the sources for our traceroutes. For each 
of these five ASes, the script will schedule traceroutes to all of the guard IPs 
found by parsing the consensuses for March 2016 above. The script sleeps for 
15 minutes when we approach RIPE Atlas's rate limits. Here's how to use it:
\begin{lstlisting}
python do_udp_traceroutes.py < guardIPSet_Mar_2016.txt > mIDs.txt
\end{lstlisting}

\subsubsection{Traceroutes from exit ASes to Destinations}
\emph{Source of all the following traceroutes:} One RIPE Atlas probe in each of the ASes of the 
exit relays in the consensus files for March 2016. (Please note that RIPE Atlas does not 
have probes in \emph{all} of the exit relay ASes.)

As mentioned earlier, use {\tt getGuardsAndExitsFromConsensusMonthDir.py} to get a 
list of all of the exit relays' IP addresses for March 2016. 
Then use {\tt getASSetFromIPList.py} 
in order to convert those IP addresses into a pickle set 
of their corresponding ASes. Here's an example:
\begin{lstlisting}
python getASSetFromIPList.py exit_AS_set_Mar_2016 ipasn_20160729.dat < exitIPSet_Mar_2016.txt
\end{lstlisting}
{\tt ipasn\textunderscore 20160729.dat} is an IP/ASN lookup database that you can build 
by following the instructions at \url{https://pypi.python.org/pypi/pyasn}.
\path{exit_AS_set_Mar_2016} will contain your output set.

RIPE Atlas allows you to schedule traceroutes to one destination IP address 
like 8.8.8.8 from multiple different source ASes (e.g., 200 source ASes), and these 
traceroutes will all fall under one RIPE Atlas measurement ID (vs. 200 IDs, which 
you could have if you scheduled all of the traceroutes separately), which is very 
convenient. However, when you go to schedule such a measurement, if just one of your 
source ASes does not have a RIPE Atlas probe in it (because as we mentioned above, 
RIPE Atlas does not have probes in all of the exit relays' autonomous systems),
the entire measurement will not be 
scheduled, and you can't tell which AS was the culprit. Thus, we made a script 
called \path{getProbeASSetFromRIPE.py} that outputs a pickle set with the 
ASes that we believe have active probes based on the output from RIPE Atlas's 
official command-line tool. Here's how we used that tool:
\begin{lstlisting}
~/.local/bin/ripe-atlas probe-search --all > allprobes.txt
\end{lstlisting}
Then we supply \path{allprobes.txt} to \path{getProbeASSetFromRIPE.py} which 
parses it. We then use \path{get_covered_ripe_ASes_set.py}
to find the intersection 
between RIPE Atlas's active AS set and our desired exit relay AS set, and 
we use that intersection for our traceroutes.

Recall that to measure the paths from exit relays to DNS resolvers, we 
emulated a number of different exit relay DNS configurations.
% We used 
%the RIPE Atlas measurement platform to obtain our traceroutes.
\begin{itemize}

\item \emph{ISP DNS:} No additional traceroutes were necessary for this experiment. 

\item \emph{Google DNS:} 

\emph{Destination of traceroutes:} 8.8.8.8 (Google's public DNS resolver)

Run the following command:
\begin{lstlisting}
python do_udp_traceroutes_from_exit_ases.py exit_AS_set_pickle_file 8.8.8.8 > 8888_mIDs.txt
\end{lstlisting}
{\tt exit\textunderscore AS\textunderscore set\textunderscore pickle\textunderscore file} 
is a pickle file with the set of all exit relays ASes. 
{\tt 8888\textunderscore mIDs.txt} is an output text file that will contain IDs that RIPE Atlas 
assigns to your measurements, and it'll contain ``True'' if a measurement 
was scheduled or ``False'' if it was not. Make sure you check it at the end 
to make sure that the measurements were carried out successfully. If they were 
not, then you'll have to redo them. Measurements can be unsuccessful if you picked 
an AS that doesn't contain a RIPE Atlas probe or if you've gone over your 
measurement limits. See the RIPE Atlas website for the rate limits 
(\url{https://atlas.ripe.net/docs/udm/\#rate-limits}).

\item \emph{Local DNS:}

\emph{Destination of traceroutes:} The IP addresses of the name servers on the delegation 
paths for resolving www.facebook.com, www.instagram.com, www.google.com, 
mail.google.com, docs.google.com, calendar.google.com, www.startpage.com, 
www.twitter.com, and www.ixquick.com. 

We modified {\tt ddptr.py} to create {\tt dd.py} and get 
the name servers. With these IP addresses in hand, you can use 
\path{do_udp_traceroutes_from_exit_ases_ddptr.py} in order 
to schedule your traceroutes. Here's an example of a command we ran:
\begin{lstlisting}
python do_udp_traceroutes_from_exit_ases_ddptr.py < facebook > facebook_ddptr_mIDs.txt
\end{lstlisting}
where ``facebook'' is a text file that contains
\begin{lstlisting}
192.5.5.241
192.33.14.30
69.171.239.12
\end{lstlisting}
This script doesn't contain any rate limiting because you won't reach the limits.

\item \emph{Status quo:}

\emph{Destination of traceroutes:} The current resolvers of the exit relays based on 
the exitmap tool. 

The mapping tool was used to provide a mapping from exit relays to the resolvers they used 
during March 2016. Some exit relays had several resolver IP addresses associated with them, 
so we randomly assigned one IP address to those relays and worked with that. In the 
mapping file, those relays that had their own IP address listed as the resolver are those 
that performed their own DNS resolution. Use 
{\tt map\textunderscore exit\textunderscore ip\textunderscore 
to\textunderscore one\textunderscore resolver\textunderscore randomly.py} to output a 
pickle dictionary where the key is the exit relay's IP address as a string and the 
value is its randomly chosen resolver's IP address as a string. Use {\tt group\textunderscore 
resolvers\textunderscore by\textunderscore dest\textunderscore n\textunderscore 
convert\textunderscore exit\textunderscore ips\textunderscore to\textunderscore ASes.py} 
in order to make efficient use of RIPE Atlas. This script will produce a pickle 
dictionary where the key is the resolver IP address and the value is the set of 
exit relay ASes that need to perform traceroutes to that resolver IP address. This 
script ignores exit relays that do their own resolution and those that use 
8.8.8.8 as their resolver because these traceroutes are covered by our other 
traceroutes for the different configurations above. Finally, use 
{\tt do\textunderscore udp\textunderscore traceroutes.py} in the 
{\tt forward\textunderscore exit\textunderscore ases\textunderscore 
to\textunderscore 3rd\textunderscore party\textunderscore resolvers} 
directory to perform the RIPE Atlas 
traceroutes according to the contents of the pickle dictionary file that you 
just made. 
Here's how to use it:
\begin{lstlisting}
python do_udp_traceroutes.py resolver_ip_to_exit_AS_set_dict_pickle_fname ripe_as_set_pickle_path > third_party_mIDs.txt
\end{lstlisting}
This script also takes in a pickle set ({\tt ripe\textunderscore as\textunderscore 
set\textunderscore pickle\textunderscore path}) 
that contains the ASes that 
currently have active RIPE Atlas probes in them so that your measurements won't 
fail due to attempting to schedule a traceroute from an exit AS that doesn't contain 
a RIPE Atlas probe. This script sleeps for 20 minutes when we start reaching 
RIPE Atlas's rate limits. The script outputs a text file that provides you with 
RIPE Atlas's measurement IDs for your traceroutes and whether they were 
successfully scheduled or not. Here's an example of an output file:
\begin{lstlisting}
True
{u'measurements': [4485078]}
True
{u'measurements': [4485079]}
False
{u'error': {u'status': 400, u'code': 104, u'detail': u'Executing this measurement request would violate your maximum daily spending limit of {max} credits.  Please stop some of your currently running measurements and try again.', u'title': u'Bad Request'}}
\end{lstlisting}

\end{itemize}

Looking back on this traceroute process, the first thing to do 
would really be to run TorPS and \emph{then}
do your traceroutes based on the guard IPs and exit IPs that your simulations 
actually use! That way, you don't end up performing unnecessary traceroutes. 
(Note: We only consider the forward direction for all of our 
traceroutes for now.)

\subsubsection{Processing the traceroutes:}
We use RIPE Atlas's Sagan library\footnote{Available online at
\url{https://github.com/RIPE-NCC/ripe.atlas.sagan}.} to parse the traceroute results. Use the 
{\tt processAtlasTraceroutes.py} script under the \path{traceroutes} directory 
to parse all of your traceroutes. The purpose of this script is to produce 
Python dictionaries where the key is the source AS of the traceroute as an 
integer and the value is another dictionary where the key is the traceroute's 
destination IP address as a string and the value is the set of ASNs as ints that 
are along the path from that source AS to the destination IP address. 

This script takes in as input the name 
you'd like to give your output dictionary pickle file, your IP/ASN lookup database, 
and the RIPE Atlas measurement IDs associated with your traceroutes. It'll output 
the pickle dictionary and a text version of that dictionary.

Here's an example of a command you might run:
\begin{lstlisting}
python processAtlasTraceroutes.py exit_ases_to_8888 ipasn_20160729.dat < 8888_mIDs_only.txt > dict_text.txt
\end{lstlisting}

In order to get a list of isolated, new-line separated measurement IDs from your result 
output files from running your traceroutes above, use {\tt mIDs\textunderscore 
to\textunderscore text\textunderscore file.py} to get a text file that you can 
input to {\tt processAtlasTraceroutes.py} above.

\subsection{How we translate IP addresses to ASes}
We use the {\tt pyasn} library to translate IP addresses to ASes. The instructions 
on their website are straightforward.

\subsection{Finding Intersections}
After you've collected all of your traceroutes, the next step is to find ASes that 
appear on both the ingress and egress Tor AS paths. We treat the ingress AS path 
as a set of ASes and the egress AS path as a set of ASes. The ASes that appear 
in the intersection of these sets are the ASes that can potentially compromise 
the anonymity of Tor users. Note: If there are missing sets, we assume that 
there was no compromise for that particular circuit.

\subsubsection{Finding Intersections for ISP DNS}
Use {\tt intersect\textunderscore isps.py} to get the intersections (compromising ASes) for 
this exit relay configuration. {\tt command\textunderscore 100k.sh} in the {\tt isps} 
directory contains the commands we used. This script takes as input your pickle dictionary of 
guard AS sets from your traceroutes, the Tor client AS of interest, the name of your 
output pickle 
dictionary, and your IP/ASN lookup database. It also takes in your TorPS simulation 
text file. It outputs a pickle dictionary with the intersections it found and also 
a text version of it. 

Starting from the beginning of the TorPS simulation text file, 
the script will pick out the guard IP address and the exit IP address that 
Tor chose for that particular circuit. It will use the guard IP address to find 
the ingress AS path set in the input dictionary your provided. It will take the 
exit IP address, convert it to an AS, use that AS to create a set of one, and 
look for an intersection between the ingress set and that set of one. It will 
output your results.

\subsubsection{Finding Intersections for Google DNS}
Use {\tt intersect\textunderscore 8888.py} to get the intersections for 
this exit relay configuration. {\tt command\textunderscore 100k.sh} in the
\path{8888} directory contains the commands we used. This script takes the same inputs as the ISP 
DNS script above 
plus a pickle dictionary of the egress AS paths from the exit relay ASes to 8.8.8.8. 
This time the exit IP address from the TorPS simulation will be converted to an 
AS and the exit AS will be used as a key 
into that egress dictionary to find the appropriate AS path set for that exit relay's 
path to 8.8.8.8. The script will output your results.

\subsubsection{Finding Intersections for Local DNS}
Use {\tt intersect\textunderscore ddptr.py} to get the intersections for this 
exit relay configuration. {\tt command\textunderscore 100k.sh} in the {\tt ``ddptr''} 
directory contains the commands we used. This script takes the same inputs as the ISP 
DNS script above plus nine pickle dictionaries of the egress AS paths from the 
exit relay ASes to mail.google.com, www.google.com, calendar.google.com, 
docs.google.com, www.facebook.com, www.instagram.com, www.ixquick.com, www.twitter.com, 
and www.startpage.com. The exit IP address from the TorPS simulation will be used 
to identify which of the nine dictionaries should be used for the egress path, and 
it will be converted to an AS and used as a key into the correct dictionary to get 
the AS path. The script will output your results.

\subsubsection{Finding Intersections for Status Quo}
Use \path{intersect_whole_enchilada.py} to get the 
intersections for our appromixation of the ``status quo'' Tor exit relay 
configurations. \path{command_100k.sh} in the directory
\path{status_quo} contains the commands we used: 

\begin{lstlisting}[basicstyle=\scriptsize\ttfamily]
#!/bin/bash

python intersect_whole_enchilada.py 
../../../traceroutes/clients_to_all_guards_forward/results/clients_to_all_guards_forward_dict.pickle 
../../../traceroutes/exit_ases_to_third_party_status_quo/results/exit_ases_to_third_party.pickle 
../../../../fresh/fresh_measurements/forward_exit_ases_to_3rd_party_resolvers/final_exit_ip_to_one_resolver_list_dict.pickle 
../../../traceroutes/exit_ases_to_8888/results/exit_ases_to_8888.pickle 
../../../traceroutes/exit_ases_to_calendar_google/ddptr/results/exit_ases_to_calendar_google_ddptr.pickle 
../../../traceroutes/exit_ases_to_docs_google/ddptr/results/exit_ases_to_docs_google_ddptr.pickle 
../../../traceroutes/exit_ases_to_facebook/ddptr/results/exit_ases_to_facebook_ddptr.pickle 
../../../traceroutes/exit_ases_to_google/ddptr/results/exit_ases_to_google_ddptr.pickle 
../../../traceroutes/exit_ases_to_instagram/ddptr/results/exit_ases_to_instagram_ddptr.pickle 
../../../traceroutes/exit_ases_to_ixquick/ddptr/results/exit_ases_to_ixquick_ddptr.pickle 
../../../traceroutes/exit_ases_to_mail_google/ddptr/results/exit_ases_to_mail_google_ddptr.pickle 
../../../traceroutes/exit_ases_to_startpage/ddptr/results/exit_ases_to_startpage_ddptr.pickle 
../../../traceroutes/exit_ases_to_twitter/ddptr/results/exit_ases_to_twitter_ddptr.pickle 
compromises_1guard_100K_7922 7922 ../../../../fresh/forPyASN/ipasn_20160729.dat 
< ../../torps_typical_my_final_model_1_guard_100000_samples.txt 
> output_dict_1guard_100K_7922.txt
. . .
\end{lstlisting}
This script 
takes the same inputs as the Local DNS script above plus the 8.8.8.8 egress 
pickle dictionary, the status quo egress pickle dictionary, and the pickle 
dictionary you produced in the traceroutes phase when you mapped a resolver to an 
exit relay. The script will use the exit relay IP address to figure out what type of 
resolution should be used. If, according to the mapping, an exit relay does its own 
resolution, the script will use the Local DNS dictionary for the egress path. If, 
according to the mapping, the exit relay uses 8.8.8.8 for resolution, the script will 
use our 8.8.8.8 dictionary for the egress path. If, according to the mapping, the exit 
relay uses another resolver, the script will use the status quo egress dictionary. For 
all of these dictionaries, the exit relay's IP address will be converted to an AS before 
being used as a key into those dictionaries. The script will output your results.

\subsection{Getting the Results}
Use {\tt command.sh} under the directories 
\path{numbers/AS_Numbers_March_2016/}\emph{X}
to get your results, where \emph{X} represents the five client AS numbers. 
Here's the {\tt command.sh} for client AS 2856:
\begin{lstlisting}
#!/bin/bash

# 8888 ##########
python ../../../../fresh/numbers/number_of_compromised_streams_per_sample.py 
../../../intersections/AS_Analysis_March_2016/8888/compromises_1guard_100K_2856.pickle 
> 8888_comp_streams_2856.txt

python ../../../../fresh/numbers/time_to_first_compromise_per_sample_only_for_comp.py 
../../../intersections/AS_Analysis_March_2016/8888/compromises_1guard_100K_2856.pickle 
> 8888_time_first_comp_2856.txt

python ../../../../fresh/numbers/popular_ASes_or_IXPs.py 
../../../intersections/AS_Analysis_March_2016/8888/compromises_1guard_100K_2856.pickle 
> 8888_culprits_2856.txt


# ddptr ##########
python ../../../../fresh/numbers/number_of_compromised_streams_per_sample.py 
../../../intersections/AS_Analysis_March_2016/ddptr/compromises_1guard_100K_2856.pickle 
> ddptr_comp_streams_2856.txt

python ../../../../fresh/numbers/time_to_first_compromise_per_sample_only_for_comp.py 
../../../intersections/AS_Analysis_March_2016/ddptr/compromises_1guard_100K_2856.pickle 
> ddptr_time_first_comp_2856.txt

python ../../../../fresh/numbers/popular_ASes_or_IXPs.py 
../../../intersections/AS_Analysis_March_2016/ddptr/compromises_1guard_100K_2856.pickle 
> ddptr_culprits_2856.txt


# isps ##########
python ../../../../fresh/numbers/number_of_compromised_streams_per_sample.py 
../../../intersections/AS_Analysis_March_2016/isps/compromises_1guard_100K_2856.pickle 
> isps_comp_streams_2856.txt

python ../../../../fresh/numbers/time_to_first_compromise_per_sample_only_for_comp.py 
../../../intersections/AS_Analysis_March_2016/isps/compromises_1guard_100K_2856.pickle 
> isps_time_first_comp_2856.txt

python ../../../../fresh/numbers/popular_ASes_or_IXPs.py 
../../../intersections/AS_Analysis_March_2016/isps/compromises_1guard_100K_2856.pickle 
> isps_culprits_2856.txt


# status_quo ##########
python ../../../../fresh/numbers/number_of_compromised_streams_per_sample.py 
../../../intersections/AS_Analysis_March_2016/status_quo/compromises_1guard_100K_2856.pickle 
> status_quo_comp_streams_2856.txt

python ../../../../fresh/numbers/time_to_first_compromise_per_sample_only_for_comp.py 
../../../intersections/AS_Analysis_March_2016/status_quo/compromises_1guard_100K_2856.pickle 
> status_quo_time_first_comp_2856.txt

python ../../../../fresh/numbers/popular_ASes_or_IXPs.py 
../../../intersections/AS_Analysis_March_2016/status_quo/compromises_1guard_100K_2856.pickle 
> status_quo_culprits_2856.txt
\end{lstlisting}

The results are 
the number of compromised streams per sample, the time-to-first-compromise, and a count 
of the compromising ASes so you can see which ASes were the most popular ``culprits.'' 

Doing {\tt wc -l} on your time-to-first-compromise files will reveal the number 
of samples out of 100,000 that were compromised. For our graphs, we set the 
time-to-first-compromise number for the samples that were uncompromised to 
April 1, 2016 as a placeholder as further explained in our paper.

\printbibliography

\end{document}
