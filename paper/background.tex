\section{Background}
\label{sec:background}
We now give a brief introduction to the pillars our attack is resting
on---website fingerprinting and the process of DNS resolution over the Tor
network.

\subsection{Website fingerprinting attacks}
The Tor network encrypts relayed traffic as it travels from the client to the
exit relay.  Therefore, intermediate parties such as the user's Internet service
provider (ISP) are not able to read any packet content.  However, Tor does not
protect meta information such as data timing, frequency, and length.  Exploiting
these properties, the ISP can use a classifier to guess what sites the user is
visiting---despite Tor's encryption.  The literature calls this kind of attack a
\emph{website fingerprinting attack}.

Tor---and any other anonymity network---could eliminate website fingerprinting
attacks by employing leak-resistant, constant-rate channels between a Tor client
and its guard.  Unfortunately, the Tor network's limited spare capacity does not
allow for such an expensive defense, but some research has worked on making it
cheaper~\cite{Cai2014a}.

\subsection{How Tor handles DNS}
Tor clients must send DNS requests over Tor to prevent DNS leakage.  Tor
does not transport UDP traffic, but it implements a workaround to ensure that
DNS requests can go over Tor.

After the user types in a domain, say foo.com, Tor Browser establish a
connection to the SOCKS proxy exposed by the local Tor client.  Using the SOCKS
protocol, applications instruct the Tor client to establish a circuit to a given
domain and port.\footnote{SOCKS in version 4a and 5 supports connection
initiations using domain names in addition to IP addresses.} The Tor client then
selects an exit relay whose exit policy supports foo.com and port 443.  Next,
the client sends a \texttt{RELAY\_BEGIN} Tor cell to the exit relay, instructing
it to first resolve foo.com, and then establish a TCP connection to the resolved
address at port 443~\cite[\S~6.2]{tor-spec}.  After successfully establishing a
connection, the exit relay responds with a \texttt{RELAY\_CONNECTED} cell to the
local Tor client.  From then on, data can be exchanged with the intended
destination.  The \texttt{RELAY\_RESOLVE} cell supports pure name resolution,
without establishing a subsequent TCP connection.  The exit relay responds with
a \texttt{RELAY\_RESOLVED} cell.

As of December 2015, exit relays resolve domain asynchronously and both the exit
relay and the client maintain a caching layer around the resolution code to
speed up repeated lookups.  Exit relays send their DNS requests to the system
resolver, which is in \texttt{/etc/resolv.conf} on Linux systems.  The system
resolver is not modified by Tor, and contains whatever the exit relay operator
configured, e.g., the ISP's resolver, or public resolvers such as 8.8.8.8.
