\section{Background}
\label{sec:background}
We provide an introduction introduction to website fingerprinting
attacks, as well as how
the Tor network implements DNS resolution.

\paragraph{Website fingerprinting attacks}
The Tor network encrypts relayed traffic as it travels from the client
to the exit relay.  Therefore, intermediate parties such as the user's
Internet service provider (ISP) cannot read the contents of any packet.
Tor does not, however, protect other statistics about the network
traffic, such as packet inter-arrival timing, directions, and frequency.
The ISP can analyze these properties to infer the destinations that a
user is visiting.  The literature calls this attack \emph{website
  fingerprinting}.

Past work evaluated website fingerprinting attacks in two settings:
A {\em closed-world} setting consists of a set of $n$ {monitored}
websites, and the attackers tries to learn which among all $n$ sites the
user is visiting with the notable restriction that the user can only
browse to one of the $n$ websites.  The {\em open-world} setting is more
realistic: the user can browse to {unmonitored} sites in addition to the
monitored sites. Unmonitored sites are, per definition, not known to the
attacker; thus, the attacker's traffic classifier cannot train on unmonitored
sites the user visits. Note that the attacker's classifier can train on
whatever unmonitored sites it chooses, as long as during testing, the
classifier has not trained on an unmonitored site. Two relevant metrics
in the open-world
setting are recall and precision.  Recall measures the probability that
a visit to a monitored page will be detected, while precision measures
the probability that a classification by the classifier of a visit to a
monitored site (positive test outcome) is the correct one. Consider a
classifier with 0.25 recall and 0.5 precision: on average, every fourth
visit by the user to a monitored site will be detected, and half of the
classifications by the classifier will be wrong. Errors can
either classify a monitored site as
unmonitored (lowering recall) or vice versa (lowering precision).
Mistaking one monitored site for another is less likely~\cite{WangThesis}.

Wa-kNN is a website fingerprinting attack by Wang et al.~\cite{Wang2014a}
in the form of a k-nearest neighbour classifier with a custom weight-learning
algorithm WLLCC~\cite{WangThesis}.  From \emph{packet traces}
between a Tor client and its guard, Wa-kNN extracts a number of \emph{features}
for \emph{site classification}.  Useful features are, e.g.,
the number of outgoing packets and bursts of packets in the same direction.
In the training phase, WLLCC adjusts \emph{weights} of features extracted from
sites of known classes such that the \emph{distance} between instances of the
same site (class) are minimized (collapsed).
In the testing phase, Wa-kNN determines the distance of a testing traffic trace
to all known training traces.  The distance calculation results in the $k$
nearest classes: if all classes are the same, then the testing trace is
classified as that class, otherwise it is classified as unmonitored.
In the open-world setting, one class represents all unmonitored sites both
during training and testing.  By increasing $k$, Wa-kNN can trade decreased
recall for increased precision.  We set $k=2$ when using Wa-kNN for higher
recall since \name is a highly precise attack.

Tor could eliminate website fingerprinting
attacks with encrypted, constant-bitrate channels between a Tor client and
its guard; other anonymity networks could use a similar technique.
Unfortunately, the Tor network's limited spare capacity does not
allow for such a throughput-intensive defense, but some research has
worked on making this type of defense
more efficient~\cite{Cai2014a,DBLP:journals/corr/JuarezIPDW15,WangThesis,adapativepadding}.

\paragraph{The Tor network}
The Tor network is an overlay network that anonymizes TCP streams such as web
traffic.  As of April 2016, it comprises approximately 7,000 relays and about two
million users.  Clients send data over the Tor network by randomly selecting
three relays---typically called the guard, middle, and exit relay---that then
form a virtual tunnel called a \emph{circuit}.  The guard relay learns the
client's IP address, but not its web activity, while the exit relay gets to
learn the client's web activity, but not its IP address.  Relays and clients
talk to each other using the Tor protocol, which uses 512-byte \emph{cells}.
Finally, each relay is uniquely identified by its \emph{fingerprint}, a hash
function over its public key.

\paragraph{How Tor resolves DNS requests}
Tor clients must send DNS requests over Tor to prevent DNS {\em leakage}
(\ie, having a
DNS request travel over an unencrypted channel as opposed to over Tor
itself).  Tor does not transport UDP
traffic, but it implements a workaround to wrap DNS requests into Tor cells.
After the user types in a domain, say {\tt foo.com}, the Tor Browser establishes a
connection to the SOCKS proxy exposed by the local Tor client.  Using the SOCKS
protocol, applications instruct the Tor client to establish a circuit to a given
domain and port.\footnote{SOCKS versions 4a and 5 support connection
initiation using domain names in addition to IP addresses.} The Tor client then
selects an exit relay whose exit policy supports foo.com and port 443.  Next,
the client sends a \texttt{RELAY\_BEGIN} Tor cell to the exit relay, instructing
it to first resolve foo.com, and then establish a TCP connection to the resolved
address at port 443~\cite[\S~6.2]{tor-spec}.  After successfully establishing a
connection, the exit relay responds with a \texttt{RELAY\_CONNECTED}
cell.  The client can then exchange data with its intended
destination.  The \texttt{RELAY\_RESOLVE} cell supports pure name resolution,
without establishing a subsequent TCP connection~\cite[\S~6.4]{tor-spec}.  The
exit relay responds with a \texttt{RELAY\_RESOLVED} cell.

As of December 2015, exit relays maintain a caching layer
around the DNS resolution code to speed up repeated lookups.
The caching layer for Tor clients is off by default to prevent tracking attacks
due to modified DNS responses~\cite{nolocalcache}.
Exit relays send
their DNS requests to the system resolver, which Linux systems read from
\texttt{/etc/resolv.conf}.  Tor does not modify the system resolver and
contains whatever the exit relay operator configured, such as the ISP's resolver,
or public resolvers such as Google's {\tt 8.8.8.8} open public DNS resolver.
