\section{Related work}
\label{sec:related_work}
Our work combines traffic correlation with website fingerprinting attacks,
which is why we split related work into these two topics.

\paragraph{Traffic analysis methods}
Tor's threat model excludes global adversaries~\cite{Dingledine2004a}, but the
practical threat of such adversaries is an important question that academia has
spent considerable effort on answering.  In 2004, when the Tor network counted
only 33 relays, Feamster and Dingledine investigated the practical threat that
network-level adversaries pose to anonymity networks~\cite{Feamster2004a}.  In
particular, the authors considered an attacker that controls an autonomous
system that is traversed both for ingress and egress traffic, allowing the
attacker to correlate both streams.  Using AS path prediction~\cite{Gao2001a},
Feamster and Dingledine found that powerful tier-1 ISPs reduce location
diversity of anonymity networks.  In 2007, Murdoch and Zieli\'{n}ski drew
attention to IXP-level adversaries, a class of adversaries that was missing in
Feamster and Dingledine's work~\cite{Murdoch2007a}.  Murdoch and Zieli\'{n}ski
showed that IXP adversaries are able to correlate traffic streams, even in the
presence of packet sampling rates as low as one in 2,000.  In 2013, Johnson
\ea~\cite{Johnson2013a} presented the first large-scale study on the risk of Tor
users facing relay-level and network-level adversaries.  The authors developed a
Tor path simulator (TorPS~\cite{TorPS}) that simulates Tor circuits for a number
of user models the authors developed.  By combining TorPS with AS path
prediction, Johnson \ea could answer questions such as the average time until a
Tor user's circuit is deanonymized by an AS or IXP.  Most recently in 2016,
Nithyanand \ea~\cite{Nithyanand2016a} used AS path prediction to evaluate the
practical threat faced by users in the top 10 countries using Tor.  In 2015,
Juen \ea~\cite{Juen2015a} questioned the accuracy of path prediction algorithms
that prior work~\cite{Johnson2013a,Feamster2004a} used to estimate the threat of
correlation attacks.  The authors compared AS path predictions to millions of
traceroutes they initiated from 25\% of Tor relays by bandwidth at the AS level.
Only 20\% of predicted paths matched the paths observed in traceroute, calling
into question the results of prior work.  A limitation of Juen \ea's work is
that they could not consider the reverse path in traceroutes.  This shortcoming
was addressed in 2015 by Sun \ea~\cite{Sun2015a}.  While past work treated
routing as static, Sun \ea leveraged the dynamic nature of routing to show that
network adversaries are a bigger threat than thought.

We improve on previous work in two significant ways; (\emph{i}) we are the first
to consider the DNS protocol for traffic analysis and evaluate its practical
threat, and (\emph{ii}) we develop and deploy a method to scale the measurement
method proposed by Juen \ea~\cite{Juen2015a}.  Our method leverages the
volunteer-run RIPE Atlas measurement platform~\cite{atlas} instead of convincing
relay operators to run third-party scripts.  This approach allows us to fully
automate our method and achieve previously unprecedented scale.

\paragraph{Website fingerprinting}
In 2009, Hermann \ea~\cite{Hermann2009a} demonstrated the first website
fingerprinting (WF) attack against anonymity systems---including Tor---in a
closed-world setting.  In 2011, Panchenko et al.~\cite{Panchenko2011a} greatly
improved on Hermann \ea's detection rate and provided insight into an open-world
setting.  In 2012, Cai et al.~\cite{Cai2012a} improved on prior work by
proposing an attack that used Hidden Markov Models to determine if a sequence of
page requests all come from the same site.  The authors used an open-world
setting for their evaluation.  Wang and Goldberg~\cite{Wang2013a} proposed an
improved attack that employed a new method for data gathering.  In 2014, Wang
\ea~\cite{Wang2014a} further improved on their results with a
k-nearest neighbour classifier Wa-kNN and a custom weightlearning algorithm
(WLLCC~\cite{WangThesis}) that in several rounds determine the optimal weights
for features extracted from traffic traces.
Cai \ea~\cite{Cai2014b}
analyzed what traffic features provide the most predictive power to detect
websites, proved a lower bound of any defense that achieves a certain level of
security, and provided a framework to investigate the performance of
WF attacks.  Juarez~\cite{Juarez2014a} critically evaluated past
attacks, showing that they all made numerous simplifying
assumptions.  The authors suggest that attacks are still
difficult to run outside a lab setting as an attacker will have to consider
operating system differences, page changes, and background traffic.
Recently, in 2016, Wang and Goldberg addressed a number of practical
roadblocks such as noisy data and maintaining a training set,
further highlighting the need for WF defenses in Tor~\cite{taoianreally}.
Panchenko et al.~\cite{Panchenko2016a} showed that
web\emph{page} fingerprinting (\ie, fingerprinting of any page on a site) is
significantly harder than web\emph{site} fingerprinting (\ie,
fingerprinting of only the start page of a site).
Hayes and Danezis proposed k-fingerprinting, an attack with
notably better performance than Wa-kNN even in the face of
WF defenses \cite{kfingerprinting}. Their attack retains 30\% accuracy in a
closed-world setting against the WTF-PAD defense by
Juarez et al.~\cite{DBLP:journals/corr/JuarezIPDW15}---a prime candidate for
deployment in Tor\footnote{\url{https://gitweb.torproject.org/torspec.git/tree/proposals/254-padding-negotiation.txt}}---at
the cost of 50\% bandwidth overhead.

We show how to correlate and use observed DNS requests with WF attacks,
resulting in WF+DNS attacks, greatly
improving precision for web\emph{site} fingerprinting.
Our two proposed WF+DNS attacks have implications for the design of WF defenses:
to mitigate WF+DNS attacks open-world evaluations of the WF defense should
minimise recall even when the WF attack is tuned to sacrifice precision for
recall.
