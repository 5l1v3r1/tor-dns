\section{Related work}
\label{sec:related_work}

\begin{table*}[t]
	\begin{tabular}{lp{7cm}}
	\toprule
	\textbf{Paper} & \textbf{Focus} \\
	\midrule
	Johnson et al.~\cite{Johnson2013a}, 2013 & Set of 372 unique IP addresses for
	various kinds of activity, e.g., web search, IRC, and BitTorrent. \\
	Murdoch et al.~\cite{Murdoch2007a}, 2007 & Protocol-agnostic, focusing on a
	sampled packets flowing between source and destination. \\
	Murdoch et al.~\cite{Murdoch2005a}, 2005 & Foo \\
	\bottomrule
	\end{tabular}
	\caption{An overview of related work on its focus.}
	\label{tab:related-focus}
\end{table*}

We focus on related work that covers low-latency anonymity networks.

As mentioned earlier, Tor's threat model excludes the ``global passive adversary'' 
and instead assumes an adversary that can launch both passive and active attacks 
on ``some fraction of network traffic''~\cite{dingledine2004tor}. They differentiate between traffic 
\textit{confirmation} attacks and traffic \textit{analysis} attacks. They state, ``Rather than focusing on these 
\textit{traffic confirmation} attacks, we aim to prevent \textit{traffic analysis} attacks, 
where the adversary uses traffic patterns to learn which points in the network 
he should attack''~\cite{dingledine2004tor}.
The authors provide examples of traffic confirmation attacks which 
include end-to-end size correlation (e.g., packet counting) and end-to-end timing correlation, 
which are ``effective in confirming endpoints of a stream''~\cite{dingledine2004tor}. They provide website 
fingerprinting as an example of a traffic analysis attack, and they mention that it 
``may be less effective against Tor, since streams are multiplexed within the same circuit, 
and fingerprinting will be limited to the granularity of cells (currently 512 bytes)''~\cite{dingledine2004tor}.

Serjantov and Sewell discussed packet counting attacks for low-latency networks
in 2003~\cite{Serjantov2003a}.  They focused on single nodes in an anonymity
network and conjectured that their technique is harder to apply to an anonymity
network.

In 2004, Feamster and Dingledine~\cite{Feamster2004a} investigated Tor's and
Mixmaster's topological diversity against an adversary that controls an
autonomous system.  The authors found that tier-1 ISPs and path asymmetry reduce
are among the major issues, reducing location diversity.  Tor's current
safeguard to select relays in disjoint /16 networks is not sufficient to defend
against large autonomous systems.

Murdoch and Zieli\'{n}ski pointed out that Feamster and Dingledine's work did
not consider Internet exchange points (IXPs)~\cite{Murdoch2007a}.  The authors
showed that IXPs are a practical threat---even if only sampled traffic is used
for traffic analysis.

Johnson et al.~\cite{Johnson2013a} simulated different types of network
activity, including web, IRC, and BitTorrent.  We only focus on web activity,
and investigate it in much more detail than the authors.

\paragraph{Traffic analysis methods}
Serjantov and Sewell proposed a simple flow correlation approach based on
counting packets that are sent to and from an anonymity node in a small time
interval~\cite{Serjantov2003a}.

\O{}verlier and Syverson~\cite{Overlier2006a} used the packet counting approach
described by Serjantov and Sewell~\cite{Serjantov2003a}.

\cite{Goldberg2010a}

\cite{Juen2015a}

\paragraph{Countermeasures}
\cite{Edman2009a}
\cite{Nithyanand2016a}
\cite{Akhoondi2012a}


\cite{Mathewson2004a}
\cite{Mittal2011a}
\cite{Wacek2013a}
\cite{Johnson2013a}
\cite{Juen2015a}
\cite{Danezis2004a}
\cite{Levine2004a}
\cite{Bauer2007a}
\cite{Dingledine2009a}


Johnson et al.~\cite{Johnson2015a} proposed TAPS, a trust-aware path selection
client for Tor relays.

\cite{torstinks}

\paragraph{Website fingerprinting}
\cite{Juarez2014a}
Showed that many variables are ignored that have large impact on classification
and running WF system is expensive.

Most recently, Panchenko et al. argued that 
\cite{Panchenko2016a}
