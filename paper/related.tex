\section{Related Work}
\label{sec:related_work}

This paper combines traffic analysis methods for correlation attacks
with website fingerprinting attacks; we discuss related work in each of
these two areas.

\paragraph{Traffic analysis and correlation attacks}
Tor's threat model excludes global adversaries~\cite{dingledine2004a}, but the
practical threat of such adversaries is an important question that the
research community has
spent considerable effort on answering.  In 2004, when the Tor network comprised
only 33 relays, Feamster and Dingledine investigated the practical threat that
network-level adversaries pose to anonymity networks~\cite{Feamster2004a}.
The authors considered an attacker that controls an autonomous
system that is traversed both for ingress and egress traffic, allowing the
attacker to correlate both streams.  Using AS path prediction~\cite{Gao2001a},
Feamster and Dingledine found that powerful tier-1 ISPs reduce location
diversity of anonymity networks.  In 2007, Murdoch and Zieli\'{n}ski drew
attention to IXP-level adversaries, a class of adversaries that was missing in
Feamster and Dingledine's work~\cite{Murdoch2007a}.  Murdoch and Zieli\'{n}ski
showed that IXP adversaries are able to correlate traffic streams, even in the
presence of packet sampling rates as low as one in 2,000.

In 2013, Johnson
\ea~\cite{Johnson2013a} presented the first large-scale study on the risk of Tor
users facing relay-level and network-level adversaries.  The authors developed a
{\em Tor path simulator (TorPS)}~\cite{TorPS} that simulates Tor circuits for a number
of user models the authors developed.  By combining TorPS with AS path
prediction, Johnson \ea could answer questions such as the average time until a
Tor user's circuit is deanonymized by an AS or IXP.  Most recently in 2016,
Nithyanand \ea~\cite{Nithyanand2016a} used AS path prediction to evaluate the
practical threat faced by users in the top 10 countries using Tor.  In 2015,
Juen \ea~\cite{Juen2015a} examined the accuracy of path prediction algorithms
that prior work~\cite{Johnson2013a,Feamster2004a} used to estimate the threat of
correlation attacks.  The authors compared AS path predictions to millions of
traceroutes they initiated from 25\% of Tor relays by bandwidth at the
AS level and found that
only 20\% of predicted paths matched the paths observed in traceroute.
Juen \ea could not consider the reverse path in traceroutes.  In 2015,
Sun \ea~\cite{Sun2015a} addressed this shortcoming; although past work treated
routing as static, Sun \ea showed that the dynamic nature of Internet
routing makes network-level adversaries stronger than previous work had considered.

We improve on previous work in two significant ways: (\emph{i}) we are
the first to evaluate how DNS traffic exacerbates traffic correlation
attacks, both in concept and in practice; and (\emph{ii}) we develop and
deploy a scalable, sustainable version of the measurement method proposed by Juen
\ea~\cite{Juen2015a}.  Our method uses the volunteer-run RIPE Atlas
measurement platform~\cite{atlas}, as opposed to relying on relay operators
to run third-party scripts.  This approach allows us to fully automate
our method and achieve previously unprecedented scale.

\paragraph{Website fingerprinting}
In 2009, Hermann \ea~\cite{Hermann2009a} demonstrated the first website
fingerprinting attack against anonymity systems---including Tor---in a
closed-world setting.  In 2011, Panchenko \ea~\cite{Panchenko2011a} greatly
improved on Hermann \ea's detection rate and provided insight into an open-world
setting.  In 2012, Cai \ea~\cite{Cai2012a} improved on previous work by
proposing an attack that used Hidden Markov Models to determine whether a sequence of
page requests all come from the same site.  The authors used an open-world
setting for their evaluation.  Wang and Goldberg~\cite{Wang2013a} proposed an
improved attack that employed a new method for data gathering.  In 2014, Wang
\ea~\cite{Wang2014a} further improved on their results with a
k-nearest neighbour classifier Wa-kNN and a custom weight-learning algorithm
(WLLCC~\cite{WangThesis}) that in several rounds determine the optimal weights
for features extracted from traffic traces.
Cai \ea~\cite{Cai2014b}
determined which traffic features provide the most predictive power to detect
websites, proved a lower bound of any defense that achieves a certain level of
security, and provided a framework to investigate the performance of
website fingerprinting attacks.
Juarez~\cite{Juarez2014a} showed that all previous attacks
made several simplifying
assumptions; the work suggested that attacks are still
difficult to run outside a lab setting as an attacker will have to consider
operating system differences, page changes, and background traffic.
Recently, in 2016, Wang and Goldberg addressed many practical
roadblocks to website fingerprinting, such as noisy data and maintaining a training set,
further highlighting the need for website fingerprinting defenses in
Tor~\cite{taoianreally}.

Panchenko \ea~\cite{Panchenko2016a} showed that web\emph{page}
fingerprinting (\ie, fingerprinting of any page on a site) is
significantly harder than web\emph{site} fingerprinting (\ie,
fingerprinting of only the start page of a site).  Hayes and Danezis
proposed k-fingerprinting, an attack with notably better performance
than Wa-kNN even in the face of defenses
\cite{kfingerprinting}. Their attack retains 30\% accuracy in a
closed-world setting against the WTF-PAD defense by Juarez et
al.~\cite{DBLP:journals/corr/JuarezIPDW15}---a prime candidate for
deployment in Tor~\cite{adapativepadding}---at the cost of 50\% bandwidth
overhead.

In this paper, we show how to correlate and use observed DNS requests in
concert with website fingerprinting attacks,
which significantly
improves precision for web\emph{site} fingerprinting.
The two \name attacks  that we present in this paper have implications
for the design of website fingerprinting defenses:
to mitigate \name attacks, open-world evaluations of the website fingerprinting
defense should minimise recall even when the website fingerprinting attack is
tuned to sacrifice precision for recall.
