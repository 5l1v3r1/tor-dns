\section{Methodology}
\label{sec:methodology}

\subsection{Threat model}
We consider the following adversaries.
\begin{description}
	\item[Network-level adversary] An adversary that monitors at least
		one autonomous system, e.g., ISP, VPS provider, government.
	\item[Relay-level adversary] An adversary that runs at least one Tor relay.
	\item[DNS provider] An adversary that operates the DNS resolver used
		by exit relays, e.g., Google.
\end{description}

\subsection{DNS resolver dataset}
\label{sec:dns-resolver-dataset}
\begin{itemize}
	\item Resolve domain under our control over all exit relays.
	\item Record requester's IP address on authoritative DNS server.
\end{itemize}

\subsection{Traceroute dataset}
\label{sec:traceroute-dataset}
\begin{itemize}
	\item Find machines that are topologically close to DNS resolvers
		used by exit relays.
	\begin{itemize}
		\item RIPE Atlas probes.
		\item Virtual private systems.
		\item PlanetLab nodes.
		\item Ask exit operators to run traceroutes for us.
	\end{itemize}
	\item Run traceroutes to DNS servers to determine path coverage.
	\item Determine ``AS inflation factor.''
\end{itemize}

\subsection{DNS root dataset}
\label{sec:dns-root-dataset}
\begin{itemize}
	\item Learn what kind of domains are resolved by exit relays to learn where
		we should traceroute to.
	\item Minimize risks.
	\begin{itemize}
		\item Get rid of timestamps if at all possible, otherwise the dataset
			could be used to deanonymize people after publishing it.  Nick W.
			said that these could be useful to learn about caching, though.
	\end{itemize}
\end{itemize}

\subsection{DNS packet sizes at entry guard}
\begin{itemize}
	\item Send many DNS requests over entry guard.
	\item Capture them on the wire and look at them.
	\item What's the best way to filter out the noise?
\end{itemize}
