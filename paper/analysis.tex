\section{Analysis}
\label{sec:analysis}

\subsection{Path inflation caused by DNS}
\begin{itemize}
	\item How many more ASes do we traverse when we consider DNS requests in
		addition to TCP connections?
	\item Demonstrate ``path inflation'' by tracerouting to most popular web
		sites that are accessed over Tor (see \S~\ref{sec:dns-root-dataset}).
	\item Reverse path might inflate AS coverage even more, but hard to measure.
\end{itemize}

\subsection{Feasability of DNS request linking}
\begin{itemize}
	\item How hard is it for a passive adversary that sees both ends to link DNS
		requests?
\end{itemize}

\subsection{Resolver query quality}
We want to learn how many exit relays use poorly configured resolvers that are
vulnerable to off-path poisoning attacks.  These relays are not only a security
but also an anonymity threat because if an adversary manages to poison the
resolver's cache, she can redirect Tor users to her own Web server, enabling
end-to-end correlation attacks.

\begin{table}[t]
	\centering
	\begin{tabular}{l r r r}
	\toprule
	\textbf{Type} & \textbf{Min.} & \textbf{Median} & \textbf{Max.} \\
	\midrule
	0x20 encoding & 0.9757 & 0.9840 & 0.9930 \\
	Random ports & 0.9947 & 0.9974 & 1.0000 \\
	\bottomrule
	\end{tabular}
	\caption{Summary statistics for many days worth of DNS experiments.  The
	majority of exit relay resolvers employs both 0x20 encoding and random
	source ports.}
	\label{tab:traversed-ass}
\end{table}
